% Options for packages loaded elsewhere
\PassOptionsToPackage{unicode}{hyperref}
\PassOptionsToPackage{hyphens}{url}
%
\documentclass[
]{book}
\usepackage{amsmath,amssymb}
\usepackage{lmodern}
\usepackage{iftex}
\ifPDFTeX
  \usepackage[T1]{fontenc}
  \usepackage[utf8]{inputenc}
  \usepackage{textcomp} % provide euro and other symbols
\else % if luatex or xetex
  \usepackage{unicode-math}
  \defaultfontfeatures{Scale=MatchLowercase}
  \defaultfontfeatures[\rmfamily]{Ligatures=TeX,Scale=1}
\fi
% Use upquote if available, for straight quotes in verbatim environments
\IfFileExists{upquote.sty}{\usepackage{upquote}}{}
\IfFileExists{microtype.sty}{% use microtype if available
  \usepackage[]{microtype}
  \UseMicrotypeSet[protrusion]{basicmath} % disable protrusion for tt fonts
}{}
\makeatletter
\@ifundefined{KOMAClassName}{% if non-KOMA class
  \IfFileExists{parskip.sty}{%
    \usepackage{parskip}
  }{% else
    \setlength{\parindent}{0pt}
    \setlength{\parskip}{6pt plus 2pt minus 1pt}}
}{% if KOMA class
  \KOMAoptions{parskip=half}}
\makeatother
\usepackage{xcolor}
\usepackage{longtable,booktabs,array}
\usepackage{calc} % for calculating minipage widths
% Correct order of tables after \paragraph or \subparagraph
\usepackage{etoolbox}
\makeatletter
\patchcmd\longtable{\par}{\if@noskipsec\mbox{}\fi\par}{}{}
\makeatother
% Allow footnotes in longtable head/foot
\IfFileExists{footnotehyper.sty}{\usepackage{footnotehyper}}{\usepackage{footnote}}
\makesavenoteenv{longtable}
\usepackage{graphicx}
\makeatletter
\def\maxwidth{\ifdim\Gin@nat@width>\linewidth\linewidth\else\Gin@nat@width\fi}
\def\maxheight{\ifdim\Gin@nat@height>\textheight\textheight\else\Gin@nat@height\fi}
\makeatother
% Scale images if necessary, so that they will not overflow the page
% margins by default, and it is still possible to overwrite the defaults
% using explicit options in \includegraphics[width, height, ...]{}
\setkeys{Gin}{width=\maxwidth,height=\maxheight,keepaspectratio}
% Set default figure placement to htbp
\makeatletter
\def\fps@figure{htbp}
\makeatother
\setlength{\emergencystretch}{3em} % prevent overfull lines
\providecommand{\tightlist}{%
  \setlength{\itemsep}{0pt}\setlength{\parskip}{0pt}}
\setcounter{secnumdepth}{5}
\usepackage{booktabs}
\usepackage{amsthm}
\makeatletter
\def\thm@space@setup{%
  \thm@preskip=8pt plus 2pt minus 4pt
  \thm@postskip=\thm@preskip
}
\makeatother
\ifLuaTeX
  \usepackage{selnolig}  % disable illegal ligatures
\fi
\IfFileExists{bookmark.sty}{\usepackage{bookmark}}{\usepackage{hyperref}}
\IfFileExists{xurl.sty}{\usepackage{xurl}}{} % add URL line breaks if available
\urlstyle{same} % disable monospaced font for URLs
\hypersetup{
  pdftitle={Analisis Resiko},
  pdfauthor={Bakti Siregar, M.Sc},
  hidelinks,
  pdfcreator={LaTeX via pandoc}}

\title{Analisis Resiko}
\author{Bakti Siregar, M.Sc}
\date{2023-05-22}

\begin{document}
\maketitle

{
\setcounter{tocdepth}{1}
\tableofcontents
}
\hypertarget{kata-pengantar}{%
\chapter*{Kata Pengantar}\label{kata-pengantar}}
\addcontentsline{toc}{chapter}{Kata Pengantar}

\hypertarget{deskripsi-buku}{%
\section*{Deskripsi Buku}\label{deskripsi-buku}}
\addcontentsline{toc}{section}{Deskripsi Buku}

\textbf{Analisa Resiko} adalah buku yang interaktif, online, dan tersedia secara gratis.

\begin{itemize}
\tightlist
\item
  Versi online berisi banyak objek interaktif (kuis, demonstrasi komputer, grafik interaktif, video, dan sejenisnya) yang dapat dipergunakan untuk menunjang \emph{pembelajaran lebih baik}.
\item
  Sebagian besar isi dari buku ini tersedia untuk \emph{dibaca offline} dalam format pdf dan EPUB.
\item
  Direncanakan akan tersedia dalam berbagai bahasa.
\end{itemize}

\hypertarget{petunjuk-penggunaan}{%
\subsection*{Petunjuk Penggunaan}\label{petunjuk-penggunaan}}
\addcontentsline{toc}{subsection}{Petunjuk Penggunaan}

Buku ini dapat dipergunakan dalam pembelajaran kurikulum aktuaria di seluruh dunia. Adapun cakupan pembelajarannya adalah analisa data kerugian dari berbagai organisasi aktuaria ternama didunia. Sehingga, buku ini cocok digunakan ditingkat universitas maupun pembelajar mandiri yang ingin lulus ujian aktuaria profesional. Selain itu, buku juga akan sangat berguna dalam pengembangan profesional berkelanjutan bagi para aktuaris maupun profesional lainnya di bidang asuransi dan industri terkait manajemen risiko keuangan.

\hypertarget{manfaat}{%
\subsection*{Manfaat}\label{manfaat}}
\addcontentsline{toc}{subsection}{Manfaat}

Salah satu manfaat penting dari buku online ini adalah pemerataan akses pengetahuan, sehingga memungkinkan masyarakat yang lebih luas untuk belajar tentang profesi aktuaria. Selain itu, setiap orang memiliki kapasitas untuk melibatkan banyak pihak melalui pembelajaran aktif yang memperdalam proses pembelajaran, menghasilkan analis terbaik dalam melakukan pekerjaan aktuaria yang solid.

Sekarang, pertanyaan besarnya adalah ``Mengapa buku ini baik untuk mahasiswa dan dosen serta orang lain yang terlibat dalam proses pembelajaran?'' Biaya adalah salah satu faktor yang sering disebut sebagai kendala utama bagi mahasiswa dan dosen dalam pemilihan buku teks. Selain itu, Mahasiswa sekarang ini lebih menyukai buku yang dapat dibawa secara secara elektronik (online).

\hypertarget{mengapa-analisa-resiko}{%
\subsection*{Mengapa Analisa Resiko?}\label{mengapa-analisa-resiko}}
\addcontentsline{toc}{subsection}{Mengapa Analisa Resiko?}

Tujuannya adalah agar buku ini pada akhirnya akan dapat dikembangkan secara serius kurikulum aktuaria. Mengingat perubahan era digital seperti sekarang ini akhirnya mendorong para aktuaris dalam melakukan analisa bergantung pada data yang dimiliki. Ide di balik nama \emph{Analisa Resiko} adalah untuk mengintegrasikan model data kerugian klasik dari probabilitas yang diterapkan dengan alat analitik modern. Secara khusus, penulis menyadari bahwa big data (termasuk media sosial dan asuransi berbasis penggunaan) akan terus berkembang dan komputasi berkecepatan tinggi sudah tersedia.

\hypertarget{ucapan-terima-kasih}{%
\section*{Ucapan Terima Kasih}\label{ucapan-terima-kasih}}
\addcontentsline{toc}{section}{Ucapan Terima Kasih}

Kami juga ingin mengucapkan terima kasih yang sebesar-sebesar pada semua pihak yang terlibat dalam pengembangan buku ini, yakni; mahasiswa-i, dosen, dan Universitas Matana atas dukungan dalam upaya bersama kami untuk menyediakan konten pendidikan dalam bidang aktuaria.

\includegraphics[width=0.5\textwidth,height=\textheight]{images/Logo.png}

\hypertarget{kontributor}{%
\section*{Kontributor}\label{kontributor}}
\addcontentsline{toc}{section}{Kontributor}

Sebagian besar dari isi buku ini diadopsi dari \textbf{\href{https://openacttexts.github.io/Loss-Data-Analytics/ChapIntro.html}{Loss Data Analytics}}. Berikut ini adalah nama-nama dan biografi singkat para penulis:

\begin{itemize}
\tightlist
\item
  \textbf{Bakti Siregar, M.Sc} adalah Kepala Program Studi dan Dosen di Jurusan Statistika Universitas Matana. Beliau juga seorang dosen yang juga bekerja sebagai ilmuwan data lepas yang memiliki antusiasme untuk analitik data besar, pembelajaran mesin, Pemodelan, dan pemecahan masalah. Orang menganggap saya programmer Matematika karena saya memiliki kemampuan yang kuat dalam program Statistik seperti R Studio, dan Python, dan juga akrab dengan alat basis data seperti MySQL dan sistem data besar baik Spark maupun Hadoop. Selain itu, saya dapat mengoperasikan salah satu perangkat lunak analitik bisnis yang paling kuat seperti Tableau.
\end{itemize}

\begin{itemize}
\tightlist
\item
  \textbf{Yosia} adalah salah satu mahasiwa terbaik di jurusan Statistika Universitas Matana. Dia juga memiliki minat dalam pembelajaran sains data dan akuturia khususnya melakukan komputasi dengan menggunakan R dan Python. Yosia bercita-cita suatu saat nanti akan menjadi seseorang yang ahli dibidang aktuaria maupun sain data. Yosia adalah mahasiswi jurusan Statistik di Universitas Matana. Dia memiliki minat teoretis yang luas serta minat dalam komputasi, ia juga sudah pernah terlibat dalam menerbitkan di jurnal Pengabdian Kepada Masyarakat \textbf{(PKM)}. Dia juga aktif dalam berbagai aktifitas organisasi kampus.
\end{itemize}

\begin{itemize}
\tightlist
\item
  \textbf{Clara Della} adalah mahasiswi jurusan Statistik di Universitas Matana. Dia memiliki minat teoretis yang luas serta minat dalam komputasi, ia juga sudah pernah terlibat dalam menerbitkan di jurnal Pengabdian Kepada Masyarakat \textbf{(PKM)}. Dia juga aktif dalam berbagai aktifitas organisasi kampus. adalah mahasiswi jurusan Statistik di Universitas Matana. Dia memiliki minat teoretis yang luas serta minat dalam komputasi, ia juga sudah pernah terlibat dalam menerbitkan di jurnal Pengabdian Kepada Masyarakat \textbf{(PKM)}. Dia juga aktif dalam berbagai aktifitas organisasi kampus.
\end{itemize}

\begin{itemize}
\tightlist
\item
  \textbf{Karen} adalah mahasiswi jurusan Statistik di Universitas Matana yang memiliki keahlian penelitiannya dengan menggunakan teori pemodelan, manajemen risiko, dan optimasi. adalah mahasiswi jurusan Statistik di Universitas Matana. Dia memiliki minat teoretis yang luas serta minat dalam komputasi, ia juga sudah pernah terlibat dalam menerbitkan di jurnal Pengabdian Kepada Masyarakat \textbf{(PKM)}. Dia juga aktif dalam berbagai aktifitas organisasi kampus.
\end{itemize}

\begin{itemize}
\tightlist
\item
  \textbf{Brigita} adalah dosen senior di Macquarie University di Australia, di mana ia menjabat sebagai direktur program sarjana aktuaria sejak 2018. Ia memperoleh gelar Ph.D.~pada tahun 2015 dari Nanyang Technological University di Singapura. Dia adalah seorang aktuaris yang berkualifikasi penuh, memegang kredensial dari US Society of Actuaries dan Australian Actuaries Institute. Minat penelitian utamanya adalah pemodelan kematian, manajemen risiko umur panjang, dan sistem bonus-malus.
\end{itemize}

\begin{itemize}
\tightlist
\item
  \textbf{Naufal} adalah seorang profesor di Universitas Matana. Dia memiliki gelar di bidang Matematika dan Ph.D.~dalam Sains: Matematika, diperoleh di University of Antwerp. Selama Ph.D., ia berhasil mengambil Magister Asuransi dan Magister Teknik Keuangan dan Aktuaria, keduanya di KU Leuven. Penelitiannya berfokus pada adaptasi dan penerapan metode statistik yang kuat untuk data asuransi dan keuangan. adalah mahasiswi jurusan Statistik di Universitas Matana. Dia memiliki minat teoretis yang luas serta minat dalam komputasi, ia juga sudah pernah terlibat dalam menerbitkan di jurnal Pengabdian Kepada Masyarakat \textbf{(PKM)}. Dia juga aktif dalam berbagai aktifitas organisasi kampus.
\end{itemize}

\begin{itemize}
\tightlist
\item
  \textbf{Garry} adalah Associate Professor di Departemen Manajemen Risiko, Asuransi, dan Kesehatan di Fox School of Business, Temple University? Dia adalah Associate dari Society of Actuaries. Dia mengajar mata kuliah Ilmu Aktuaria dan Manajemen Risiko di tingkat sarjana dan pascasarjana. Minat penelitiannya meliputi tata kelola perusahaan asuransi, manajemen modal, dan analisis sentimen. Dia menerima gelar Ph.D.~dari The Wharton School of the University of Pennsylvania.
\end{itemize}

\hypertarget{kritik-saran}{%
\section*{Kritik \& Saran}\label{kritik-saran}}
\addcontentsline{toc}{section}{Kritik \& Saran}

Buku teks interaktif yang tersedia secara gratis mewakili usaha baru dalam pendidikan aktuaria dan kami membutuhkan masukan Anda. Meskipun banyak upaya telah dilakukan untuk pengembangan, kami mengharapkan cegukan. Harap beri tahu instruktur Anda tentang peluang untuk peningkatan, hubungi kami melalui situs proyek kami, atau hubungi kontributor bab secara langsung dengan saran peningkatan.

Berikut ini dilampirkan beberapa peninjau atau pembaca yang telah memberikan saran dan pendapat mengenai pengembangan buku ini, adalah:

\begin{itemize}
\tightlist
\item
  mahasiswa 1
\item
  mahasiswa 2
\item
  mahasiswa 3
\item
  mahasiswa 4
\item
  mahasiswa 5
\item
  mahasiswa 6
\end{itemize}

\hypertarget{introduction-to-loss-data-analytics}{%
\chapter{Introduction to Loss Data Analytics}\label{introduction-to-loss-data-analytics}}

Chapter Preview. This book introduces readers to methods of analyzing insurance data. Section 1.1 begins with a discussion of why the use of data is important in the insurance industry. Section 1.2 gives a general overview of the purposes of analyzing insurance data which is reinforced in the Section 1.3 case study. Naturally, there is a huge gap between the broad goals summarized in the overview and a case study application; this gap is covered through the methods and techniques of data analysis covered in the rest of the text.

\hypertarget{frequency-modeling}{%
\chapter{Frequency Modeling}\label{frequency-modeling}}

\hypertarget{goodness-of-fit}{%
\section{Goodness of Fit}\label{goodness-of-fit}}

\hypertarget{modeling-loss-severity}{%
\chapter{Modeling Loss Severity}\label{modeling-loss-severity}}

\hypertarget{modifikasi-pertanggungan}{%
\section{modifikasi pertanggungan}\label{modifikasi-pertanggungan}}

Coverage modifications atau modifikasi pertanggungan adalah perubahan yang dibuat pada syarat dan ketentuan polis asuransi. Perubahan ini dapat diprakarsai oleh pemegang polis atau perusahaan asuransi, dan dirancang untuk mengubah pertanggungan yang diberikan oleh polis.

Modifikasi pertanggungan dapat dilakukan karena berbagai alasan. Sebagai contoh, pemegang polis mungkin ingin meningkatkan batas pertanggungan pada polis mereka untuk melindungi diri mereka sendiri dari potensi kerugian. Atau, mereka mungkin ingin menambah atau menghapus jenis pertanggungan tertentu, seperti menambahkan asuransi banjir pada polis pemilik rumah atau menghapus pertanggungan tabrakan dari polis mobil.

pada bagian ini membahas mengenai

\hypertarget{policy-deductibles}{%
\subsection{policy deductibles}\label{policy-deductibles}}

pada polis deductible biasa, pemegang polis setuju untuk menanggung sejumlah klaim asuransi sebelum perusahaan asuransi membayarkan klaim. Sehingga bagian kerugian yang ditanggung dan menjadi tanggung jawab pemegang polis untuk membayar deductible dengan uang mereka sendiri.

sebagai contoh jika sebuah polis memiliki daductible sebesar Rp.500 dan pemegang polis mengalami kerugian dengan biaya sebesar Rp.2500, maka perusahaan asuransi hanya akan membayar Rp.2000 (yaitu total biaya perbaikan dikurangi deductible Rp.500)

deductible sendiri di notasikan dengan \(d\), maka

\begin{itemize}
\item
  jika kerugian melebihi \(d\) atau nilai deductible, maka perusahaan asuransi bertanggung jawab untuk menanggung total kerugian dikurangin dengan deductible atau \(d\)
\item
  tergantung dengan perjanjiannya, deductible dapat berlaku untuk setiap kerugian atau total dari seluruh kerugian.
\end{itemize}

jumlah dari deductible biasanya dipilih pada saat pemegang polis membeli polis dan disesuaikan dengan kebutuhannya selama masa berlaku polis. deductible yang lebih tinggi akan menghasilkan pembayaran premi yang lebih rendah, dikarenakan pemegang polis menanggung lebih banyak saat terjadinya kerugian.

lalu jika \(X\) di notasikan sebagai kerugian yang diterima oleh pemegang polis dan \(Y\) dinotasikan sebagai jumlah klaim yang dibayarkan oleh perusahaan asuransi, maka ada dua variabel berdasarkan pembayarannya kepada pemegang polis.
a. pembayaran per kerugian
b. pembayaran per pembayaran

pada variabel perbayaran per kerugian, dinotasikan sebagai \(Y^L\) atau \((X-d)_+\) atau left censor, atau ketika jumlah atau total kerugian yang dialami kurang dari deductible, maka dinilai sama dengan 0 atau tidak dilakukan pembayaran. maka variabel ini didefinisikan sebagai

\[
Y^{L} = \left( X - d \right)_{+} 
= \left\{ \begin{array}{cc}
0 & X \le d, \\
X - d & X > d  
\end{array} \right. .
\]

disisi lain, variabel pembayaran per pembayaran dinotasikan sebagai \(Y^P\) didefinisikan ketika hanya terjadinya pembayaran, terutama \(Y^P\) sama dengan \(X-d\) dengan syarat \([X>d]\), atau dinotasikan sebagai \(Y^P=X-d||X>d\) atau dituliskan sebagai

\[
Y^{P} = \left\{ \begin{matrix}
\text{Undefined} & X \le d \\
X - d & X > d .
\end{matrix}  \right.
\]

disini \(Y^P\) disebut juga sebagai left truncated atau variabel kerugian berlebih, karena klaim yang lebih kecil dari \(d\) tidak dilaporkan dan nilai dari \(d\) berubah sebesar \(d\)

ketika nilai distribusi dari nilai kerugian bersifat kontinu, namun distribusi dari \(Y^L\) adalah gabungan kombinasi dari komponen nilai diskrit dan kontinu. bagian diskrit terletak pada \(Y=0\) atau saat \((X \leq d)\) dan komponen nilai kontinu terletak pada interval \(Y>0\) atau saat \(X>d\)

\hypertarget{policy-limit}{%
\subsection{2. Policy Limit}\label{policy-limit}}

policy limit atau batas polis adalah bentuk jumlah maksimum yang dibayarkan oleh perusahaan asuransi untuk pertanggungan tertentu berdasarkan polis asuransinya. sehingga kerugian yang ditanggung dinotasikan sebagai \(X\), dan batas pertanggunannya atau batas polisnya dinotasikan sebagai \(u\), jika kerugian melebihi batas polis \(X-u\) harus dibayar oleh pemegang polis sendiri. batas polis yang lebih tinggi berarti premi yang dibayar oleh pemegang polis semakin besar.

sebagai contoh sebuah polis mungkin memiliki batas pertanggungan sebesar Rp100.000 per kejadian, yang berarti bahwa perusahaan asuransi tidak akan membayar lebih dari Rp100.000 untuk setiap klaim atau tanggung jawab yang menjadi bagian dari polis.
dimana biaya kerugian pemegang polis dinotasikan sebagai \(X\) dan klaim yang dibayarkan oleh perusahaan asuransi dinotasikan sebagai \(Y\), dan variabel policy limit dinotasikan sebagai \(X \land u\). atau disebut sebagai right censored variable dikarenakan nilai dari \(u\) di set sama dengan \(u\). maka variabel \(Y\) didefinisikan sebagai

\[
Y = X \land u = \left\{ \begin{matrix}
X & X \leq u \\
u & X > u. \\
\end{matrix} \right.\
\]

pada batas polis, perbedaan antara \(Y^L\) dan \(Y^P\) tidak dibutuhkan dikarenakan perusahaan asuransi akan selalu melakukan pembayaran.

dengan \((X-u)\) dan \((X \land u)\) maka expektasi dari pembayaran terjadi tanpa modifikasi pertangguangan \(X\). jumlah ekspektasi pembayaran dari deductible \(u\) dan limit \(u\) maka, \(X=(X-u)_++(X\land u)\)

jika kerugian merupakan subjek dari deductible \(d\) dan limit \(u\), maka didefinisikan sebagai

\[
Y^{L} = \left\{ \begin{matrix}
0 & X \leq d \\
X - d  & d <  X \leq u \\
 u - d  & X > u. \\
\end{matrix} \right.\
\]

maka, \(Y^L\) dapat dinyatakan sebagai \(Y^L=(X\land u)-(X\land u)\).

\hypertarget{policy-deductible-and-policy-limit}{%
\subsection{policy deductible and policy limit}\label{policy-deductible-and-policy-limit}}

\begin{itemize}
\item
  pada policy deductible atau pengurangan polis, jika kerugian yang dialami oleh pemegang polis kurang dari nilai deductible maka perusahaan asuransi tidak akan membayarkan kerugian tersebut, dan jika lebih besar dari nilai deductible maka klaim yang dibayarkan merupakan total dari kerugian dikurangi dengan nilai deductible, sehingga sisa biaya kerugian ditanggung pemegang polis.
  semakin besar nilai deductible maka besar premi yang perlu dibayarkan oleh pemegang polis semakin rendah
\item
  pada policy limit atau pembatasaan polis, jika kerugian yang dialami oleh pemegang polis lebih besar dari nilai limit maka perusahaan asuransi tidak akan membayarkan kerugian tersebut, dan jika masih dibawah dari batas limit maka perusahaan asuransi akan selalu membayarkan total kerugian tersebut. akan tetapi jika lebih besar dari limit sisa biaya kerugian ditanggung pemegang polis.
  semakin besar nilai limit maka besar premi yang dibayarkan oleh pemegang polis semakin besar
\end{itemize}

\hypertarget{coinsurance-and-inflation}{%
\subsection{Coinsurance and inflation}\label{coinsurance-and-inflation}}

coinsurance atau koasuransi adalah jenis pengaturan asuransi di mana dua atau lebih perusahaan asuransi berbagi risiko yang terkait dengan satu polis. Dalam pengaturan koasuransi, setiap perusahaan asuransi mengasumsikan sebagian risiko yang terkait dengan polis dan bertanggung jawab untuk membayar bagian proporsional dari setiap klaim yang muncul. Coinsurance sering digunakan pada asuransi properti dan asuransi kecelakaan, di mana besarnya risiko dapat melebihi kapasitas penanggung tunggal untuk menanggungnya.

pada Policy Deductibles jumlah kerugian yang ditanggung oleh pemegang polis sampai dengan nilai dari deductible \(d\). kerugian yang dapat ditanggung juga dapat berupa presentase dari klaim. presentase \(\alpha\) sering disebut sebagai faktor koasuransi. jika polis merupakan subjek dari deductible dan limit polis, maka koasuransi mengacu pada presentase klaim yang harus ditanggung oleh perusahaan asuransi.

setelah dilakukan deductible dan limit pada polis maka, variabel pembayaran per kerugiaan atau \(Y^L\) didefinisikan sebagai:

\[
Y^{L} = \left\{ \begin{matrix}
0 & X \leq d, \\
\alpha\left( X - d \right) & d <  X \leq u, \\
\alpha\left( u - d \right) & X > u. \\
\end{matrix} \right.\
\]

jumlah maksimum yang dapat dibayarkan oleh perusahaan asuransi adalah \(\alpha (u-d)\), dimana u adalah maksimum klaim yang dibayarkan dan pada Policy limit ketika kerugian merupakan subjek pada deductible \(d\) dan limit \(u\) untuk variabel per kerugian atau \(Y^L\), maka dapat dinyatakan sebagai \(Y^L=(X\land u)-(X\land d)\), maka pada koasuransi \(Y^L\) dapat dinyatakan sebagai\(Y^L=\alpha[(X\land u)-(X\land d)]\).

\hypertarget{reinsurance}{%
\subsection{Reinsurance}\label{reinsurance}}

Reinsurance atau Reasuransi adalah jenis asuransi yang digunakan perusahaan asuransi untuk mengalihkan sebagian risiko yang telah mereka tanggung dalam menjamin polis asuransi kepada perusahaan asuransi lain. Dalam pengaturan reasuransi, perusahaan asuransi menyerahkan sebagian risiko yang terkait dengan polis atau portofolio polis kepada perusahaan reasuransi, yang mengasumsikan risiko tersebut dengan imbalan sebagian premi yang dibayarkan oleh pemegang polis. Reasuransi biasanya digunakan oleh perusahaan asuransi untuk melindungi diri mereka sendiri dari kerugian akibat bencana atau untuk mengelola eksposur mereka terhadap risiko di lini bisnis tertentu.

pada Policy deductible berdasarkan polis tersebut, pemegang polis harus membayar semua kerugian hingga batas nilai deductible, dan perusahaan asuransi hanya membayar jumlah (jika ada) di atas batas nilai deductible. terdapat peraturan dimana di mana perusahaan asuransi mengalihkan sebagian risiko polis dengan mendapatkan pertanggungan dari perusahaan asuransi lain dengan membayarkan juga premi asuransi.

Reasuransi adalah pengaturan kontrak di mana perusahaan asuransi mengalihkan sebagian dari risiko yang diasuransikan dengan mendapatkan pertanggungan dari perusahaan asuransi lain dengan imbalan premi reasuransi. Dalam kontrak tersebut, penanggung utama atau perusahaan asuransi awal harus melakukan semua pembayaran yang diperlukan kepada pemegang polis hingga total pembayaran penanggung utama mencapai deductible reasuransi yang telah ditetapkan. lalu Perusahaan asuransi lainnya kemudian hanya bertanggung jawab untuk membayar kerugian di atas deductible reasuransi. Jumlah maksimum yang dipertahankan oleh penanggung utama dalam perjanjian reasuransi disebut retensi.

\hypertarget{coinsurance-and-reinsurance}{%
\subsection{Coinsurance and Reinsurance}\label{coinsurance-and-reinsurance}}

Perbedaan utama antara reasuransi dan koasuransi adalah arah pengalihan risiko. Dalam pengaturan reasuransi, perusahaan asuransi mengalihkan risiko kepada perusahaan asuransi lain, sedangkan dalam pengaturan koasuransi, beberapa perusahaan asuransi berbagi risiko yang terkait dengan satu polis. Selain itu, reasuransi biasanya digunakan untuk melindungi penanggung dari kerugian akibat bencana atau untuk mengelola eksposur risiko mereka, sementara koasuransi sering digunakan untuk memungkinkan penanggung menanggung polis yang lebih besar daripada yang dapat mereka tangani sendiri.

\hypertarget{model-selection-and-estimation}{%
\chapter{Model Selection and Estimation}\label{model-selection-and-estimation}}

test

\hypertarget{aggregate-loss-models}{%
\chapter{Aggregate Loss Models}\label{aggregate-loss-models}}

Sub bab ini membahas mengenai pembangunan model probabilitas untuk menggambarkan klaim agregat oleh sistem asuransi yang terjadi dalam periode waktu tertentu. Sistem asuransi dapat berupa polis tunggal, kontrak asuransi kelompok, lini bisnis , atau seluruh buku bisnis perusahaan asuransi. Dalam bab ini, klaim agregat mengacu pada jumlah klaim dari portofolio kontrak asuransi.

Pertimbangkan portofolio asuransi dari \(N\) kontrak individu, dan \(S\) menunjukkan kerugian agregat portofolio dalam jangka waktu tertentu. Ada dua pendekatan untuk memodelkan kerugian agregat \(S\) , model risiko individu dan model risiko kolektif. Model risiko individu menekankan kerugian dari masing-masing kontrak individu dan mewakili kerugian agregat sebagai:

\[S_n=X_1 +X_2 +\cdots+X_n,\]

Di mana \(X_i~(i=1,\ldots,n)\) diinterpretasikan sebagai jumlah kerugian dari \(X_i\) kontrak. \(N\) menunjukkan jumlah kontrak dalam portofolio dan dengan demikian merupakan angka tetap daripada variabel acak. Untuk model risiko individu, biasanya diasumsikan \(X_i\) ini independen. Karena fitur kontrak yang berbeda seperti cakupan dan paparan , \(X_i\) belum tentu terdistribusi secara identik. Fitur penting dari distribusi masing-masing \(X_i\) adalah massa probabilitas pada nol yang sesuai dengan peristiwa tidak adanya klai

Model risiko kolektif mewakili kerugian agregat dalam hal distribusi frekuensi dan distribusi keparahan:

\[S_N=X_1 +X_2 + \cdots + X_N .\]

Sejumlah klaim acak \(N\) yang dapat mewakili baik jumlah kerugian atau jumlah pembayaran. Sebaliknya, dalam model risiko individual biasanya menggunakan sejumlah kontrak tetap \(N\).\(X_1, X_2, \ldots, X_N\) sebagai representasi dari jumlah masing-masing kerugian. Setiap kerugian mungkin atau mungkin tidak sesuai dengan kontrak unik.

Misalnya, mungkin ada banyak klaim yang timbul dari satu kontrak. Itu wajar untuk dipikirkan \(X_i>0\) karena jika \(X_i=0\) maka tidak ada klaim yang terjadi. Biasanya kita menganggap bahwa kondisional pada \(X_{1},X_{2},\ldots ,X_{n}\) adalah iid variabel acak. Distribusi dari N dikenal sebagai distribusi frekuensi , dan distribusi umum dari \(X\) dikenal sebagai distribusi keparahan . Dengan berasumsi \(N\) Dan \(X\) sendiri. Dengan model risiko kolektif, sehingga dapat menguraikan kerugian agregat menjadi frekuensi \(( N )\) proses dan tingkat keparahan \(( X )\) model. Fleksibilitas ini memungkinkan analis untuk mengomentari dua komponen terpisah ini. Misalnya, pertumbuhan penjualan karena standar penjaminan emisi yang lebih rendah dapat menyebabkan frekuensi kerugian yang lebih tinggi tetapi mungkin tidak memengaruhi keparahan. Demikian pula, inflasi atau kekuatan ekonomi lainnya dapat berdampak pada keparahan tetapi tidak pada frekuensi.

\hypertarget{menghitung-distribusi-klaim-agregat}{%
\chapter{5.4 Menghitung Distribusi Klaim Agregat}\label{menghitung-distribusi-klaim-agregat}}

Bagian ini membahas dua pendekatan praktis untuk menghitung distribusi kerugian agregat, yaitu metode rekursif dan simulasi.

\hypertarget{metode-rekursif}{%
\section{metode rekursif}\label{metode-rekursif}}

penggunaan metode rekursif untuk membangun model majemuk dengan komponen frekuensi \(N\) yang termasuk dalam kelas \((a,b,0)\) atau \((a,b,1)\), dan komponen tingkat keparahan \(X\) yang memiliki distribusi diskrit.

Namun, jika distribusi tingkat keparahan \(X\) kontinu, praktik yang umum dilakukan adalah mendiskritkan distribusinya terlebih dahulu agar metode rekursif dapat diterapkan.

Dalam hal ini, diasumsikan bahwa N termasuk dalam kelas \((a,b,1)\), sehingga nilai probabilitas \(N\) pada waktu \(k\) dinyatakan sebagai \(pk = (a + bk) pk-1\). Selanjutnya, diasumsikan bahwa support (nilai yang mungkin) dari \(X\) terbatas pada \({0,1,...,m}\), dan distribusinya diskrit. Oleh karena itu, fungsi probabilitas dari \(S_N\) dapat dinyatakan dalam

\begin{aligned}
f_{S_N}(s)&=\Pr (S_N=s) \\
&=\frac{1}{1-af_{X}(0)}\left\{ \left[ p_1 -(a+b)p_{0}\right]
f_X (s)+\sum_{x=1}^{s\wedge m}\left( a+\frac{bx}{s} \right) f_X (x)f_{S_N}(s-x)\right\}.
\end{aligned}

Jika \(N\) berada dalam kelas \((a,b,0)\) maka \(p1 = (a + b)p0\) dan seterusnya

\begin{align*}
f_{S_N}(s)=\frac{1}{1-af_X (0)}\left\{ \sum_{x=1}^{s\wedge m}\left( a+\frac{bx
}{s}\right) f_X (x)f_{S_N}(s-x)\right\}.
\end{align*}

karena model ARIMA yang digunakan berbeda. Persamaan tersebut hanya memperhitungkan faktor skala \(a\) dan \(b\) dan mengakumulasi probabilitas dari setiap nilai \(x\) dari \(X\) hingga mencapai nilai \(s\) yang diinginkan

\hypertarget{contoh}{%
\subsection{contoh}\label{contoh}}

Jumlah klaim dalam suatu periode \(N\) memiliki distribusi geometrik dengan mean 4. Besarnya setiap klaim \(X\) mengikuti \(Pr(X=x)=0.25\), untuk \(x=1,2,3,4\). Jumlah klaim dan jumlah klaim bersifat independen. \(S_N\) adalah jumlah klaim keseluruhan pada periode tersebut.

Hitunglah \(F_{S_N}(3)\).

Solusi Distribusi tingkat keparahan \(X\) adalah sebagai berikut
\(f_X(x)=\frac14\), \(x=1,2,3,4\). Distribusi frekuensi \(N\) adalah geometris dengan rata-rata 4, yang merupakan anggota dari kelas \((a,b,0)\) dengan \(b=0\), \(a=\frac\beta{1+\beta}=\frac45\), dan \(p0=\frac1{1+\beta}=\frac15\). nilai dari komponen tingkat keparahan \(X\) adalah \({1,…,m=4}\), yang bersifat diskrit dan terbatas. Dengan demikian, kita dapat menggunakan metode rekursif

\begin{aligned}
f_{S_N} (x) &= 1 \sum_{y=1}^{x\wedge m} (a+0) f_X (y) f_{S_N} (x-y) \\
&= \frac{4}{5} \sum_{y=1}^{x\wedge m} f_X (y) f_{S_N} (x-y) .
\end{aligned}

Solusi ditemukan dengan menggunakan metode rekursif, di mana fungsi probabilitas \(f_{S_N}(x)\) untuk setiap nilai \(x\) dihitung menggunakan rumus \(f_{S_N}(x) = \sum_{y=1}^{x\wedge m} (a+0) f_X(y) f_{S_N}(x-y)\), di mana \(m=4\) adalah nilai maksimum dari distribusi nilai klaim \(X\), dan \(a=\frac{\beta}{1+\beta}=\frac{4}{5}\) dan \(p_0=\frac{1}{1+\beta}=\frac{1}{5}\) adalah parameter dari distribusi frekuensi geometri yang diberikan.

khususnya kita memiliki

\begin{aligned}
f_{S_N} (0) &= \Pr(N=0) = p_0=\frac{1}{5}\\
f_{S_N} (1) &= \frac{4}{5}\sum_{y=1}^{1} f_X (y) f_{S_N} (1-y) = \frac{4}{5} f_X(1) f_{S_N}(0)\\
&= \frac{4}{5}\left( \frac{1}{4}\right)\left(\frac{1}{5} \right) = \frac{1}{25}\\
f_{S_N} (2) &=  \frac{4}{5}\sum_{y=1}^{2} f_X (y) f_{S_N} (2-y) = \frac{4}{5} \left[ f_X(1)f_{S_N}(1) + f_X(2) f_{S_N}(0) \right] \\
&= \frac{4}{5}\left[ \frac{1}{4} \left( \frac{1}{25} + \frac{1}{5}\right) \right] =
\frac{4}{5}\left( \frac{6}{100}\right) = \frac{6}{125}\\
f_{S_N} (3) &= \frac{4}{5} \left[ f_X(1) f_{S_N}(2) + f_X(2)f_{S_N}(1) + f_X(3) f_{S_N}(0) \right]\\
&= \frac{4}{5}\left[ \frac{1}{4} \left( \frac{1}{25} + \frac{1}{5} +
\frac{6}{125}\right) \right] = \frac{1}{5}\left( \frac{5+25+6}{125}\right) = 0.0576\\
\Rightarrow \ F_{S_N} (3) &= f_{S_N} (0)+f_{S_N} (1)+f_{S_N} (2) +f_{S_N} (3) = 0.3456 .
\end{aligned}

Setelah menghitung nilai \(f_{S_N}(0)\), \(f_{S_N}(1)\), \(f_{S_N}(2)\), dan \(f_{S_N}(3)\), fungsi distribusi kumulatif \(F_{S_N}(3)\) diperoleh dengan menjumlahkan nilai-nilai tersebut. Hasil akhirnya adalah \(F_{S_N}(3) = 0.3456\).

\hypertarget{simulasi}{%
\section{simulasi}\label{simulasi}}

Distribusi kerugian agregat dapat dievaluasi dengan menggunakan simulasi Monte Carlo. Untuk kerugian agregat, Simulasi Monte Carlo digunakan untuk menghasilkan sampel acak dari kerugian keseluruhan berdasarkan distribusi probabilitas yang dianggap sesuai untuk distribusi frekuensi dan tingkat keparahan klaim.

gunanya adalah seseorang dapat menghitung distribusi empiris dari \(S_N\) dengan menggunakan sampel acak. Nilai ekspektasi dan varians dari kerugian agregat juga dapat diperkirakan dengan menggunakan rata-rata sampel dan varians sampel dari nilai simulasi.

Sekarang kita rangkum prosedur simulasi untuk model kerugian agregat. Misalkan \(m\) adalah ukuran sampel acak yang dihasilkan dari kerugian agregat.

Individual Risk Model: \$S\_n = X\_1 + ⋯ + X\_n \$

\begin{itemize}
\tightlist
\item
  misalkan \(j=1,...,m\) menjadi penghitung, dimulai dari \(j = 1\)
\item
  Hitung setiap realisasi kerugian individu \(x_{ij}\) untuk \(i=1,...,n\) . Sebagai contoh, hal ini dapat dilakukan dengan menggunakan metode transformasi invers
\item
  Hitung kerugian keseluruhan \(s_j = x_{1j} + ⋯ + x_{nj}\).
\item
  terkahir Ulangi dua langkah di atas untuk \(j=2,...,m\) untuk mendapatkan sampel berukuran \(m\) dari \(S_n\), dengan kata lain \({s_1,...,s_m}\).
\end{itemize}

Collective Risk Model : \(S_n = X_1 + ... + X_n\)

\begin{itemize}
\tightlist
\item
  misalkan \(j=1,...,m\) menjadi penghitung, dimulai dari \(j = 1\)
\item
  Hitung jumlah klaim \(n_j\) dari distribusi frekuensi \(N\).
\item
  Diberikan \(n_j\), hasilkan jumlah setiap klaim secara independen dari distribusi tingkat keparahan \(X\), dilambangkan dengan \(x_{1j},...,x_{{n_j}j}\).
\item
  Hitung kerugian keseluruhan \(s_j = x_{1j} + ⋯ + x_{{n_j}j}\).
\item
  Ulangi tiga langkah di atas untuk \(j=2,...,m\) untuk mendapatkan sampel berukuran \(m\) dari \(S_N\), dengan kata lain \({s_1,...,s_m}\)
\end{itemize}

Dengan sampel acak \(S\), distribusi empiris dapat dihitung sebagai

\begin{aligned}
\hat{F}_S(s)=\frac{1}{m}\sum_{i=1}^{m}I(s_i\leq s),
\end{aligned}

Untuk individual risk model, kerugian keseluruhan dihitung sebagai jumlah kerugian individu yang acak,

sedangkan untuk collective Risk Model, kerugian keseluruhan dihitung sebagai jumlah kerugian dari sejumlah klaim.

Dalam kedua kasus, sampel acak dihasilkan dari distribusi probabilitas yang dianggap sesuai, dan kemudian distribusi empiris dari sampel tersebut dihitung untuk memperkirakan distribusi probabilitas dari kerugian agregat.

dimana \(I(\cdot)\) adalah fungsi indikator. distribusi empiris \(\hat{F}_S(s)\) akan dikonvergen ke \({F}_S(s)\), dikarenakan ukuran sampel \(m\rightarrow \infty\)

Dalam perhitungannya, asumsi-asumsi awal dibuat tentang distribusi probabilitas dan parameter-parameternya, kemudian model-model ini diestimasi menggunakan data yang tersedia dan kualitas kecocokan model dievaluasi menggunakan alat validasi model. Proses ini memberikan cara yang berguna untuk memperkirakan risiko yang terkait dengan kerugian agregat, dan dapat membantu perusahaan atau organisasi dalam merencanakan dan mengelola risiko mereka.

Prosedur di atas mengasumsikan bahwa distribusi probabilitas, termasuk nilai parameter, dari distribusi frekuensi dan tingkat keparahan telah diketahui. Dalam praktiknya, kita perlu mengasumsikan distribusi-distribusi ini terlebih dahulu, mengestimasi parameter-parameternya dari data, dan kemudian menilai kualitas kecocokan model dengan menggunakan berbagai alat validasi model. Sebagai contoh, asumsi-asumsi dalam model risiko kolektif menyarankan estimasi dua tahap di mana satu model dikembangkan untuk jumlah klaim \(N\) dari data jumlah klaim, dan model lainnya dikembangkan untuk tingkat keparahan klaim \(X\) dari data jumlah klaim.

(edit sendiri kalo ga pas, kalimatnya di ganti juga gapapa)
tertanda -Garry

\hypertarget{simulation-and-resampling}{%
\chapter{Simulation and Resampling}\label{simulation-and-resampling}}

\hypertarget{premium-foundations}{%
\chapter{Premium Foundations}\label{premium-foundations}}

\hypertarget{pengenalan-ratemaking}{%
\chapter{7.1 Pengenalan Ratemaking}\label{pengenalan-ratemaking}}

Pada bagian ini, Anda akan belajar cara:

Menggambarkan ekspektasi sebagai metode dasar untuk menentukan premi asuransi
Menganalisis persamaan akuntansi untuk menghubungkan premi dengan kerugian, biaya, dan keuntungan
Merangkum strategi untuk memperluas penetapan harga untuk mencakup risiko heterogen dan tren dari waktu ke waktu.

Bab ini menjelaskan bagaimana menentukan harga yang tepat untuk produk asuransi, yang dikenal sebagai premi. Premi adalah jumlah uang yang dibebankan untuk perlindungan asuransi terhadap kejadian yang tidak pasti. Dalam asuransi, harga/premi ini dikenal sebagai tarif karena dinyatakan dalam unit standar, misalnya harga per seribu dolar pertanggungan atas rumah atau manfaat jika terjadi kematian.

Namun, keunikan asuransi adalah bahwa biaya perlindungan asuransi tidak diketahui pada saat penjualan kontrak. Biaya mungkin tidak terungkap selama berbulan-bulan atau bertahun-tahun, tergantung pada kejadian yang diasuransikan. Oleh karena itu, penetapan harga asuransi berbeda dengan pendekatan ekonomi pada umumnya.

Dalam pendekatan penetapan harga aktuaria tradisional, harga ditentukan sebagai fungsi dari biaya asuransi. Premi dianggap sebagai sumber pendapatan yang menyediakan pembayaran klaim, biaya kontrak, dan margin operasi, yang dapat dirumuskan dalam persamaan akuntansi:

\begin{equation}
\small{
\text{Premium = Loss + Expense + UW Profit} .
}
\end{equation}

Namun, ada pasar asuransi di mana harga aktuaria hanya memberikan masukan untuk harga pasar umum. Untuk memperkuat perbedaan ini, premi berbasis biaya aktuaria kadang-kadang dikenal sebagai harga teknis. Oleh karena itu, keputusan perusahaan seperti penetapan harga harus dievaluasi dengan mengacu pada dampaknya terhadap nilai pasar perusahaan. Tujuan ini lebih komprehensif daripada gagasan statis tentang maksimalisasi laba.

Istilah Biaya dapat dibagi menjadi biaya yang bervariasi berdasarkan premi, seperti komisi penjualan, dan yang tidak, seperti biaya bangunan dan gaji karyawan. Istilah Keuntungan UW adalah singkatan dari keuntungan underwriting dan dapat mencakup biaya modal. Persamaan ini berlaku untuk jumlah banyak kontrak, atau portofolio, dan digunakan untuk membantu menetapkan premi, seperti dengan menetapkan tujuan laba.

Istilah kerugian dalam persamaan tersebut didasarkan pada biaya yang diharapkan, karena sulit untuk memprediksi kerugian yang tepat untuk masing-masing kontrak. Namun, teks tersebut mengakui bahwa pendekatan ini mengasumsikan adanya ketidakpastian dan memperkenalkan prinsip-prinsip premi alternatif yang memasukkan ketidakpastian ke dalam penetapan harga. Bab ini juga memperluas pertimbangan penetapan harga ke kumpulan risiko yang heterogen dan membahas perkembangan dan tren pengalaman kerugian untuk mengembangkan tingkat suku bunga ke depan.

Terakhir, bab ini memperkenalkan metode untuk memilih premi dengan membandingkan metode pemeringkatan premi dengan kerugian dari portofolio yang ditahan dan memilih metode yang menghasilkan kecocokan terbaik dengan data yang ditahan. Bab ini juga mencakup suplemen teknis mengenai peraturan pemerintah tentang tarif asuransi.

\hypertarget{metode-penentuan-tarif-gabungan}{%
\chapter{7.2 Metode Penentuan Tarif Gabungan}\label{metode-penentuan-tarif-gabungan}}

Pada bagian ini, Anda akan belajar tentang:

\begin{itemize}
\tightlist
\item
  Definisi pure premium sebagai biaya kerugian serta dalam hal frekuensi dan keparahan.
\item
  Menghitung tarif yang diindikasikan menggunakan pure premiums, biaya, dan beban keuntungan.
\item
  Definisi rasio kerugian.
\item
  Menghitung perubahan tarif yang diindikasikan menggunakan rasio kerugian.
\item
  Membandingkan metode pure premium dan rasio kerugian untuk menentukan premi.
\end{itemize}

Dalam kasus ini, diasumsikan terdapat \(n\) kontrak asuransi dengan kerugian (losses) \(X1,...,Xn\). Kontrak-kontrak tersebut memiliki distribusi kerugian yang sama dan dianggap sebagai portofolio homogen yang terdiri dari kontrak-kontrak yang sama. Hal ini dapat diterapkan pada asuransi pribadi seperti asuransi mobil atau asuransi rumah di mana perusahaan asuransi menulis banyak kontrak pada risiko yang sangat mirip. Selain itu, asumsi tentang distribusi yang identik tidak terlalu membatasi karena dalam bagian selanjutnya akan diperkenalkan variabel paparan yang memungkinkan pengalaman dapat diskalakan agar dapat dibandingkan. Dalam kasus ini, diasumsikan bahwa \(X1,...,Xn\) adalah iid (independen dan identik terdistribusi).

\hypertarget{metode-penghitungan-premi-murni}{%
\section{7.2.1 Metode Penghitungan Premi Murni}\label{metode-penghitungan-premi-murni}}

Dalam metode ini, diperoleh estimasi kerugian yang diharapkan dengan menghitung rata-rata dari kerugian yang terjadi pada seluruh polis dalam suatu kumpulan (n polis).

\begin{equation}
\small{
\mathrm{E}(X) \approx \frac{\sum_{i=1}^n X_i}{n} = \frac{\text{Kerugian}}{\text{Eksposur}} = \text{Premi Murni}.
}
\end{equation}

Dalam kasus risiko homogen, di mana semua polis dianggap sama, jumlah polis n dapat digunakan sebagai ukuran eksposur. Namun, pada Bagian 7.4.1, konsep eksposur diperluas ketika polis tidak memiliki karakteristik yang sama.

Untuk mendapatkan premi murni, kita juga dapat menggunakan pendekatan frekuensi-keparahan. Dalam hal ini, premi murni dihitung sebagai hasil kali antara frekuensi klaim dan besar kerugian.

\begin{equation}
\small{
\text{Premi Murni} = \frac{\text{jumlah klaim}}{\text{Eksposur}} \times \frac{\text{Kerugian}}{\text{jumlah klaim}} = \text{frekuensi} \times \text{keparahan}.
}
\end{equation}

Ketika menggunakan metode premi murni, dapat digunakan baik rata-rata kerugian (biaya kerugian) maupun pendekatan frekuensi-keparahan untuk menentukan premi.

Untuk lebih mendekatkan diri pada aplikasi dalam praktik, sekarang kita kembali ke persamaan (7.1) yang menyertakan biaya. Persamaan (7.1) juga mengacu pada Laba UW untuk laba underwriting. Ketika diskalakan dengan premi, ini dikenal sebagai pembebanan laba. Karena klaim tidak pasti, perusahaan asuransi harus memiliki modal untuk memastikan bahwa semua klaim dibayar. Memegang modal ekstra ini adalah biaya menjalankan bisnis, investor di perusahaan perlu dikompensasi untuk ini, dengan demikian pemuatan ekstra.

Sekarang kita menguraikan Beban menjadi beban yang bervariasi berdasarkan premi, Variabel, beban yang tidak bervariasi,dan Premi Tetap, sehingga Beban = Variabel + Premi Tetap. Dengan menganggap biaya variabel dan laba sebagai bagian dari premi, kita mendefinisikan

\begin{equation}
\small{
V =  \frac{\text{Variable}}{\text{Premium}} ~~~ \text{and}~~~
Q = \frac{\text{UW Profit}}{\text{Premium}} ~.
}
\end{equation}

Dengan definisi dan persamaan (7.1) ini, kita dapat menulis

\begin{equation}
\small{
\begin{matrix}
\begin{array}{ll}
\text{Premium} &= \text{Losses + Fixed} + \text{Premium} \times \frac{\text{Variable + UW Profit}}{\text{Premium}}  \\
& = \text{Losses + Fixed} + \text{Premium} \times (V+Q) .
\end{array}
\end{matrix}
}
\end{equation}

Penyelesaian untuk hasil premi

\begin{equation}
\small{
\text{Premium} = \frac{\text{Losses + Fixed}}{1-V-Q} .
}
\end{equation}

Dibagi dengan eksposur, tarif dapat dihitung sebagai

\begin{equation}
\begin{matrix}
\begin{array}{ll}
\text{Rate} &= \frac{\text{Premium}}{\text{Exposure}} = \frac{\text{Losses/Exposure + Fixed/Exposure}}{1-V-Q} \\
&=   \frac{\text{Pure Premium + Fixed/Exposure}}{1-V-Q} ~.
\end{array}
\end{matrix}
\end{equation}

Dengan kata lain, ini adalah

\begin{equation}
\small{
\text{Rate} =\frac{\text{pure premium + fixed expense per exposure}}{\text{1 - variable expense factor - profit and contingencies factor}}  .
}
\end{equation}

\hypertarget{metode-rasio-kerugian}{%
\section{7.2.2 Metode Rasio Kerugian}\label{metode-rasio-kerugian}}

Rasio kerugian adalah rasio jumlah kerugian terhadap premi

\begin{equation}
\small{
\text{Loss Ratio} = \frac{\text{Loss}}{\text{Premium}} .
}
\end{equation}

Ketika menentukan premi, agak berlawanan dengan intuisi untuk menekankan rasio ini karena komponen premi dimasukkan ke dalam penyebut. Seperti yang akan kita lihat, metode rasio kerugian mengembangkan perubahan tingkat daripada tingkat; kita dapat menggunakan perubahan tingkat untuk memperbarui pengalaman masa lalu untuk mendapatkan tingkat saat ini. Untuk melakukan hal ini, perubahan tingkat terdiri dari rasio rasio kerugian pengalaman terhadap rasio kerugian target. Faktor penyesuaian ini kemudian diterapkan pada rate saat ini untuk mendapatkan rate yang baru.

Untuk melihat cara kerjanya dalam konteks yang sederhana, mari kita kembali ke persamaan (7.1) tetapi sekarang abaikan biaya untuk mendapatkan \$ Premi = Kerugian + Keuntungan UW \$. Membagi dengan premi menghasilkan

\begin{equation}
\small{
\frac{\text{UW Profit}}{\text{Premium}} = 1 - LR = 1 - \frac{\text{Loss}}{\text{Premium}} .
}
\end{equation}

Misalkan kita memiliki pemuatan laba ``target'' baru, katakanlah \(Q_{target}\) . Dengan asumsi bahwa kerugian, eksposur, dan hal-hal lain mengenai kontrak tetap sama, maka untuk mencapai target pemuatan laba yang baru, kita akan menyesuaikan premi. Gunakan ICF untuk faktor perubahan yang ditunjukkan yang didefinisikan melalui ekspresi

\begin{equation}
\small{
\frac{\text{New UW Profit}}{\text{Premium}} = Q_{target} =  1 - \frac{\text{Loss}}{ICF \times \text{Premium}}.
}
\end{equation}

Menyelesaikan untuk \(ICF\), kita mendapatkan

\}\begin{equation}
\small{
ICF =  \frac{\text{Loss}}{\text{Premium} \times (1-Q_{target})} = \frac{LR}{1-Q_{target}}.
}
\end{equation}

Jadi, sebagai contoh, jika kita memiliki rasio kerugian saat ini = 85\% dan target keuntungan \(Q_{target} = 0,20\), maka \(ICF = 0,85/0,80 = 1,0625\), yang berarti kita meningkatkan premi sebesar 6,25\%.

Sekarang mari kita lihat bagaimana hal ini bekerja dengan biaya dalam persamaan (7.1). Kita dapat menggunakan pengembangan yang sama seperti pada Bagian 7.2.1 dan mulai dengan persamaan (7.2), selesaikan pembebanan laba untuk mendapatkan

\begin{equation}
\small{
Q = 1 - \frac{\text{Loss+Fixed}}{\text{Premium}} - V .
}
\end{equation}

Kita menginterpretasikan kuantitas \(Rugi + Premi Tetap + V\) sebagai ``rasio biaya operasional''. Sekarang, tetapkan persentase keuntungan Q pada target dan sesuaikan premi melalui ``faktor perubahan yang ditunjukkan'' \$ ICF

\begin{equation}
\small{
Q_{target} = 1
-\frac{\text{Loss + Fixed}}{\text{Premium}\times ICF} - V .
}
\end{equation}

Menyelesaikan untuk hasil \$ ICF\$

\begin{equation}
{\small
\begin{array}{ll}
ICF &= \frac{\text{Loss + Fixed}}{\text{Premium} \times (1 - V - Q_{target})} \\
&= \frac{\text{Loss Ratio + Fixed Expense Ratio}}{1 - V - Q_{target}} .
\end{array}
}
\end{equation}

(edit sendiri kalo ga pas, kalimatnya di ganti juga gapapa)
tertanda -Garry

\hypertarget{risk-classification}{%
\chapter{Risk Classification}\label{risk-classification}}

\hypertarget{experience-rating-using-credibility-theory}{%
\chapter{Experience Rating Using Credibility Theory}\label{experience-rating-using-credibility-theory}}

\hypertarget{insurance-portfolio-management-including-reinsurance}{%
\chapter{Insurance Portfolio Management including Reinsurance}\label{insurance-portfolio-management-including-reinsurance}}

\hypertarget{loss-reserving}{%
\chapter{Loss Reserving}\label{loss-reserving}}

\hypertarget{experience-rating-using-bonus-malus}{%
\chapter{Experience Rating using Bonus-Malus}\label{experience-rating-using-bonus-malus}}

\hypertarget{aggregate-loss-models-1}{%
\chapter{Aggregate Loss Models}\label{aggregate-loss-models-1}}

\hypertarget{dependence-modeling}{%
\chapter{Dependence Modeling}\label{dependence-modeling}}

\hypertarget{appendix-a-review-of-statistical-inference}{%
\chapter{Appendix A: Review of Statistical Inference}\label{appendix-a-review-of-statistical-inference}}

\hypertarget{appendix-b-iterated-expectations}{%
\chapter{Appendix B: Iterated Expectations}\label{appendix-b-iterated-expectations}}

\hypertarget{appendix-c-maximum-likelihood-theory}{%
\chapter{Appendix C: Maximum Likelihood Theory}\label{appendix-c-maximum-likelihood-theory}}

\hypertarget{appendix-d-summary-of-distributions}{%
\chapter{Appendix D: Summary of Distributions}\label{appendix-d-summary-of-distributions}}

\hypertarget{appendix-e-conventions-for-notation}{%
\chapter{Appendix E: Conventions for Notation}\label{appendix-e-conventions-for-notation}}

\hypertarget{section}{%
\chapter*{}\label{section}}
\addcontentsline{toc}{chapter}{}

\end{document}
